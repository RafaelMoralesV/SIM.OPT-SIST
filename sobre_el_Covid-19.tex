\section{Sobre el COVID-19}
Antes de empezar a desarrollar sobre los mecanismos de modelación y simulación de la Pandemia del COVID-19, es necesario hablar más detalladamente del virus.

\subsection*{Mecanismos de Transmisión}
El principal método de transmisión para este virus consiste en el transmisión por 'gotitas'. Una persona infectada al momento de respirar o toser impulsa una cantidad de gotitas y pequeñas particulas que contienen el virus. Estas gotitas pueden y son respiradas por otras personas, o caer en ojos, nariz o boca, o incluso contaminar las superficies que estas toquen, lo que puede generar una infección en esta nueva persona previamente sana. \cite{cdc_2021}

La CDC lista principalmente 3 formas en que este virus se esparce:

\begin{itemize}
    \item "Al inhalar estando cerca de una persona infectada que exhala pequeñas gotitas y partículas respiratorias que contienen el virus.".
    \item "Al hacer que estas pequeñas gotitas y partículas respiratorias que contienen el virus se depositen sobre los ojos, nariz o boca, especialmente a través de salpicaduras y aspersiones como las generadas al toser o estornudar."
    \item "Al tocarse los ojos, la nariz o la boca con las manos contaminadas con el virus."
\end{itemize}

Este tipo de transmisión aerea ha hecho necesario el uso de mascarillas en todo el público general, además de medidas de cuarentena durante los primeros periodos de la pandemia. Junto con esto, otras medidas de prevensión de contagio que se han establecido ha sido la ventilación constante de espacios cerrados o reducidos, el uso de alcohol gel para sanitización de manos (Principalmente por el tercer punto de infección listado previamente) y jabón para el mismo propósito. 

\subsection*{Mecanismos de Detección}

\subsection*{Mutaciones y nuevas variantes}