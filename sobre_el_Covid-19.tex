\section{Sobre el COVID-19}
Antes de empezar a desarrollar sobre los mecanismos de modelación y simulación de la Pandemia del COVID-19, es necesario hablar más detalladamente del virus.

\subsection*{Mecanismos de Transmisión}
El principal método de transmisión para este virus consiste en el transmisión por 'gotitas'. Una persona infectada al momento de respirar o toser impulsa una cantidad de gotitas y pequeñas particulas que contienen el virus. Estas gotitas pueden y son respiradas por otras personas, o caer en ojos, nariz o boca, o incluso contaminar las superficies que estas toquen, lo que puede generar una infección en esta nueva persona previamente sana. \cite{cdc_2021}

La CDC lista principalmente 3 formas en que este virus se esparce:

\begin{itemize}
    \item "Al inhalar estando cerca de una persona infectada que exhala pequeñas gotitas y partículas respiratorias que contienen el virus.".
    \item "Al hacer que estas pequeñas gotitas y partículas respiratorias que contienen el virus se depositen sobre los ojos, nariz o boca, especialmente a través de salpicaduras y aspersiones como las generadas al toser o estornudar."
    \item "Al tocarse los ojos, la nariz o la boca con las manos contaminadas con el virus."
\end{itemize}

Este tipo de transmisión aerea ha hecho necesario el uso de mascarillas en todo el público general, además de medidas de cuarentena durante los primeros periodos de la pandemia. Junto con esto, otras medidas de prevensión de contagio que se han establecido ha sido la ventilación constante de espacios cerrados o reducidos, el uso de alcohol gel para sanitización de manos (Principalmente por el tercer punto de infección listado previamente) y jabón para el mismo propósito. 

\subsection*{Mecanismos de Detección}
Actualmente están en uso dos pruebas de detección del virus: Prueba de reacción en adena de la polimerasa con transcripción inversa (RT-PCR por sus siglas en inglés), y una prueba de antígenos.

\subsubsection*{Prueba de Antígenos}
Esta prubea de COVID-19 es de baja precisión pero de muy rápido uso. Consiste en la detección de ciertas proteínas del virus en el cuerpo.

Es importante notar que esta prueba se considera precisa cuando las instrucciones se siguen detenidamente, pero aún así es posible la existencia de un falso negativo, o en otras palabras, que una persona que si esté infectada con el virus produzca un resultado que indique lo contrario. Es por esta razón que el Ministerio de Salud de Chile sugiere este test rápido solo en casos de ligeras sospechas, y solo como alternativa a la disponibilidad oportuna del examen de PCR, y solo se considera un resultado negativo como tal cuando el caso de sospecha clínica es muy bajo. \cite{salud}

Este test, además, baja su rendimiento en diversos casos. Minsal, una vez más, declara que después de los 7 días de síntomas el rendimiento de este test disminuye considerablemente, y el resultado se vuelve poco confiable. Además, no todos los test de antígenos son creados por igual, por lo que no todos tienen un buen rendimiento.

El caso positivo de este test, de todas formas, es tan válido como el test de PCR. Es sólo el caso negativo el que puede ser dudado por profesionales de la salud dependiendo del contexto.

\subsubsection*{RT-PCR}
Esta prueba consiste en la detección del material genético del virus mediante una técnica de laboratorio llamada reacción en cadena de la polimerasa (PCR, por sus siglas en inglés) con transcipción inversa.

La detección del virus con esta técnica es muy precisa, pero requiere equipo especializado de laboratio. Cuando se hace de forma interna normalmente demora solo minutos, pero para la detección de personas comunes este tipo de test normalmente requiere la derivación de muestras a sitios especializados, lo que retrasa los resultados a días, o incluso semanas si este test se vuelve muy solicitado. Para el Minsal, esta es la principal forma de confirmar casos de COVID-19 como positivos.




\subsection*{Mutaciones y nuevas variantes}


