\section{conclusión}
La pandemia del COVID-19 ha sido y es hasta el día de hoy un caótico evento, muy difícil de predecir. El intento incesable de científicos y académicos a lo largo del mundo ha sido, sin embargo, crucial para poder hacer las mejores estimaciones posibles para lograr tomar las mejores decisiones en todo el mundo.

Las decisiones que los estados y gobiernos han tomado pueden no parecer acertadas cuando son analizadas en retrospectiva, sin embargo, no hay que olvidar que muchas de estas fueron creadas con poco tiempo y pocos datos con los que trabajar. En este sentido, las simulaciones y modelos matemáticos han probado ser vitales.

Las nuevas variantes de COVID, por desgracia, han demostrado que los modelos no son oráculos. Pese a ser muy acertados en el corto plazo, estas predicciones son tan susceptibles a cambios inesperados, que las hacen poco viables para el largo plazo.

Creo, sin embargo, que esta pandemia va a dar a luz a nuevos modelos más sofisticados que puedan utilizarse para cualquier emergencia a futuro que pudiera ocurrir, de características similares. Si bien lo mejor sería que ningún brote epidémico vuelva a ocurrir, es cierto que estamos más preparados que nunca.
