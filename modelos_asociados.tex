\section{Modelos Asociados}

El modelo es el resultado de la investigación e interpretación de un sistema. Se constituye como un desarrollo mediante el cual se realiza una descripción lo más detallada de la realidad, una especie de abstracción, que permite plasmar diferentes procesos, problemáticas, soluciones u otras interacciones varias que se deseen predecir en el sistema modelado.

Los modelos de simulación, en particular, nacen a partir de un modelo de sistema que trata de representar en un diseño las partes más importantes del sistema mismo, junto con una serie de objetivos que quiere lograr la simulación.

Los sistemas reales son extremadamente difíciles de replicar en un modelo de forma 100\% fiel, pues en lo común, son compuestos por cantidades casi imposibles de elementos que interactúan, muchas veces de formas altamente sutiles. Sin embargo, hay ventajas muy claras al momento de intentarlo:

\begin{itemize}
    \item No es el sistema real, por lo que las operaciones son interrumpibles y los resultados 'no son reales'.
    \item Son modificables, por lo que permiten mucha flexibilidad al momento de experimentar con distintos escenarios, políticas y otras variables.
    \item Son una abstracción que intentan entregar un resultado más sencillo (pero no menos verídico) para propósitos de estudio de estos mismos. O sea, muchas veces sus resultados son más interpretables que los del sistema real.
\end{itemize}

Así mismo, los modelos matemáticos presentan desventajas:

\begin{itemize}
    \item Dependiento del modelo, pueden caer en simplificaciones exageradas, lo que puede generar un modelo no apto para múltiples situaciones y entradas para el sistema que se pretende modelar.
    \item Como todo tipo de modelo, dependen de un estudio profundo del sistema, por lo que cualquier predicción o resultado generado del modelo es tan solo tan bueno como el entendimiento que se tiene del sistema mismo.
    \item Su implementación puede ser costosa o compleja.
\end{itemize}

\subsection*{Tipos de modelos de simulación}
Podemos nombrar un total de 7 tipos distintos de modelos de simulación

\begin{enumerate}
    \item Modelos de simulación discreta
    \item Modelos de simulación continua
    \item Modelos de simulación combinada discreta-continua
    \item Modelos de simulación determinística o estocástica
    \item Modelos de simulación estática/dinámica
    \item Modelos de simulación con orientación hacia los eventos
    \item Modelos de simulación con orientación hacia los procesos
\end{enumerate}

Para este documento, solo me centraré en los modelos de simulación discreta y los modelos de simulación continua, sin embargo, es importante tener en consideración la existencia de los demás tipos.

\subsection*{Consideraciones}
Es necesario tener en consideración los siguientes aspectos del sistema para un correcto modelado:

\begin{enumerate}
    \item Estructura del sistema: Cuales son los elementos que componen a este sistema
    \item Dinámica del sistema: Como se desarrolla y transforma el sistema cuando este cambia en el tiempo.
    \item Recursos del sistema: Que partes del sistema son compartidos
\end{enumerate}
