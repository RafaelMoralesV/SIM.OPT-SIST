\section{Modelos predictivos}

La modelación matemática del virus se ha hecho desde el inicio de la pandemia. Ha ayudado a la predicción de la transmisión del virus, y comportamiento del mismo en poblaciones concretas. Por supuesto, estas simulaciones son tan solo tan buenas como la cantidad de estudio que se ha hecho del sistema en que se basan, y en el poco tiempo inicial en que salieron los primeros papers, las predicciones se caracterizaron por ser agridulcemente optimistas al mirarlas en la situación actual de la pandemia.

\begin{figure}
    \includegraphics[width=\columnwidth]{casos estimados españa.png}
    \caption{Casos estimados en España según el modelo del día 19 de abril (día 47 de la serie de datos registrados). Fuente: Modelos Predictivos de la Epidemia de Covid-19 en españa con curvas de gompertz, por Sánchez-Villegas, Pablo y Codina, Antonio Daponte}
\end{figure}

Según uno de los primeros reportes hechos en España, la expectativa de la pandemia era de un total de 240.000 contagios y 25.000 fallecidos, con un final pronosticado para la epidemia entre junio y julio de 2020\cite{sanchez-villegas_codina_2020}. La realidad de la situación de España es de 11.8 Millones de casos, y 104.000 muertes, y a fecha de abril de 2022, no parece estar muy cercano el final de esta epidemia, con nuevas variantes apareciendo aún.

Sin embargo, por más que sea divertido ver las predicciones de modelos poco entrenados en retrospectiva, estos eventualmente si lograron llegar a predicciones más y más acertadas, y gran parte del buen desarrollo de la pandemia se debe al uso que han tenido estos modelos matemáticos para poder predecir y anticipar la forma en que el virus se transmitiría o evolucionaría.

\subsection{Modelo SIR} 
Este es un modelo clásico para epidemias, de nombre SIR, o Susceptibles. Infectados y Recuperados, creado por Kermack y McKendrick. 

Se basa en el uso de ecuaciones diferenciales ordinarias para describir una mecánica de contagios en una población cerrada de $N$ individuos susceptibles a contagiarse por el virus. A partir de un contagio inicial, este modelo describe el contagio a una determinada velocidad de infección $I$. Tras un periodo de tiempo, una persona infectada en este modelo, que no haya fallecido, se vuelve inmune al mismo; deja de recibir nuevas infecciones y pasa a ser catalogado como Recuperado $R$. Con forme pasa el tiempo en esta simulación, la población que es susceptible al contagio disminuye, hasta el punto en que esta deja de existir, resultando en una transformación total de la población inicial susceptible en población resistente y personas fallecidas.

El sustento del modelo SIR son un trio de ecuaciones diferenciales ordinarias descritas de la siguiente manera:

\begin{equation} \frac{dS}{dt} = -\beta S I \label{SIR_1} \end{equation}

Describe el cambio de personas susceptibles en un instante en concreto. Este siempre debe tender a un valor menor o igual a cero, pues se espera que conforme avance la epidemia, menos personas susceptibles queden, pues van siendo infectadas en el trascurso de esta.

\begin{equation} \frac{dI}{dt} = \beta S I - \nu I \label{SIR_2} \end{equation}

Representa cuantos nuevos infectados aparecen en un instante. Este valor siempre es mayor o igual a cero.

\begin{equation} \frac{dR}{dt} = \nu I \label{SIR_3} \end{equation}

Representa la cantidad de nuevos individuos resistentes a la infección en un instante.

Para que estas expresiones funcionen, los valores de la razón de transmición $ \beta > 0 $ y la taza de recuperación $ \nu > 0 $.
La Expresión $ \beta S I $ corresponde a la cantidad de nuevas infecciones en un determinado instante. Si reemplazamos la ecuación \eqref{SIR_3} en \eqref{SIR_2} podemos conseguir un valor para esta expresión en base a la derivada de Infectados y Recuperados tal que:

\begin{equation} \frac{dI}{dt} + \frac{dR}{dt} = \beta S I \end{equation}

La cual, a su vez, que posible reemplazar dentro de la ecuación \eqref{SIR_1}.

\begin{equation} \frac{dS}{dt} + \frac{dI}{dt} + \frac{dR}{dt} = 0 \end{equation}

Integrar esta última expresión nos da una expresión base para un modelo SIR que muestra uno de los problemas que contiene:

\begin{equation} S + I + R = Cte \end{equation}

El valor constante que se encontró con estas operaciones corresponde al numero $N$ del que se habló en un inicio. Como la población es cerrada, nunca cambia a lo largo del tiempo. El modelo SIR asume, además, que la epidemia presentada es relativamente breve en el tiempo. Tampoco ocurren nacimientos ni muertes naturales. No hay un periodo de infección latente, lo que implica que un individuo se vuelve infeccioso en el instante en que es infectado él mismo. La inmunidad que se obtiene del virus es permanente, y, el mayor problema, una mezcla en masa de individuos.

La mezcla en masa de individuos asume que la razón de encuentro entre la población susceptible e infectada es proporcional al producto de ambas poblaciones. Si se dobla la cantidad de cualquiera de estas resulta en el doble de infecciones en un instante de tiempo, lo cual es una suposición extraña. Un individuo en concreto solo mantiene contacto con una cantidad reducidad de otros individuos dentro de su propia comunidad.

Es natural razonar que la epidemia modelada por el Modelo SIR culmina en el tiempo cuando la cantidad de personas Susceptibles o la cantidad de personas infectadas llega a uno valor cero. Sin embargo, es posible comprobar que es imposible que el modelo SIR produzca una situación donde la población susceptible llegue a cero. Lo que modela la simulación es que, cuando ocurre un brote epidémico, la población susceptible decrese hasta un valor límite denotado por $S^\infty$. La población infectada, por otra parte, incrementa hasta un valor máximo, para luego decrecer hasta la extinción, comportamiento que pese a ser apto para la gran cantidad de epidemias que ha enfrentado la humanidad, se queda corto para el caso del COVID.

Este tipo de suposición poco razonables hace del modelo SIR uno poco útil para modelar la transmición del COVID-19. Sin embargo, aún es un punto inicial muy apto para la generación de otros modelos basados en éste.

\begin{figure}
    \includegraphics[width=\columnwidth]{Modelo SIR.png}
    \caption{Grafica de un Modelo SIR de un problema concreto \cite{weiss_2013}}
    \label{grafico modelo sir}
\end{figure}

En la figura \ref{grafico modelo sir} podemos apreciar una simulación calculada para un modelo matemático SIR determinado. Dentro de esta, podemos apreciar las tres ecuaciones que componen al modelo, y como estas cambiar a lo largo del tiempo.

\subsection{Modelo Matemático de SEIR}
Debido a los problemas y suposiciones del modelo SIR, múltiples variaciones han sido creadas a lo largo del tiempo para ayudar a los científicos e investigadores a simular y estudiar mejor sistemas de epidemiología. El modelo SEIR nace como uno de estos, en donde el problema en que las personas expuestas al virus se vuelven infecciosas inmediatamente. La forma de solución, es generar un nuevo estado para una persona de la población susceptible inicial llamado 'expuesto', donde una persona ya fue infectada, pero aún no es capaz de transmitir la infección a otras de la población susceptible.

Sin embargo, este modelo clásico sigue generando un par de problemas de ambiguedad y suposiciones, por lo que, para el caso del COVID-19, investigadores procuraron modificarlo incluso más, para mejor modelar la enfermedad.

\begin{figure}
    \includegraphics[width=\columnwidth]{Diagrama SEIR.png}
    \caption{Diagrama de estados de una persona en el modelo SEIR, basado en la caracterización hecha por Peng y colaboradores. Elaboración propia.}
    \label{diagrama SEIR}
\end{figure}

El modelo clásico SEIR se compone de las siguientes cuatro ecuaciones:
\begin{eqnarray}
    \frac{dS(t)}{dt} = - \beta * \frac{S(t) I(t)}{N} \label{SEIR1}\\
    \frac{dE(t)}{dt} = \beta * \frac{S(t) I(t)}{N} - \gamma E(t) \label{SEIR2}\\
    \frac{dI(t)}{dt} = \gamma E(t) - \delta I(t) \label{SEIR3}\\
    \frac{dR(t)}{dt} = \delta I(t) \label{SEIR4}
\end{eqnarray}

Este modelo de cuatro estados describe entonces, en su ecuación \eqref{SEIR1}, el cambio de población susceptible en un instante, \eqref{SEIR2}: el cambio de su población expuesta en un instante, \eqref{SEIR3}: el cambio de su población infectada en un instante, y finalmente, \eqref{SEIR4}: el cambio de su población recuperada en un instante.

Lianrong Peng, y colaboradores, describen en un documento de título "Análisis epidémico del COVID-19 en China mediante modelación dinámica"\cite{https://doi.org/10.48550/arxiv.2002.06563}, un modelo basado en el Modelo SEIR que introduce 7 estados para la población, descritos en la figura \ref{diagrama SEIR}. Este modelo modifica las ecuaciones básicas del modelo SEIR con el fin de mejor representar la epidemia a fecha de 2020. Descrito abajo se muestra los siete estados descrito por Peng y colaboradores:

\begin{equation} \label{PENG}
    \begin{split}
    \frac{dS(t)}{dt} & = - \beta * \frac{S(t) I(t)}{N} - \alpha S(t) \\
    \frac{dE(t)}{dt} & = \beta * \frac{S(t) I(t)}{N} - \gamma E(t) \\
    \frac{dI(t)}{dt} & = \gamma E(t) - \delta I(t) \\
    \frac{dQ(t)}{dt} & = \delta I(t) - \lambda(t) Q(t) - \kappa(t) Q(t) \\
    \frac{dR(t)}{dt} & = \lambda(t) Q(t) \\
    \frac{dD(t)}{dt} & = \kappa(t) Q(t) \\
    \frac{dP(t)}{dt} & = \alpha S(t)
    \end{split}
\end{equation}

En este sistema, las ecuaciones representan lo siguiente:
\begin{itemize}
    \item S(t): el número de casos susceptibles.
    \item E(t): el número de casos expuestos, aquellos que han sido infectados por el virus, pero aún no tienen la capacidad de transmitirlo ellos mismos.
    \item I(t): el número de personas que han sido infectadas, y que no han sido puestos en cuarentena.
    \item Q(t): el número de casos en cuarentena, aquellos que son positivos en la enfermedad, y son puestos en una condición que les imposibilita infectar a otros.
    \item R(t): el numero de casos recuperados o curados.
    \item D(t): el número de casos que han fallecido.
    \item P(t): el número de casos no susceptibles. Son aquellos que se han protegido o inmunizado, y ya no contraen la enfermedad.
\end{itemize}

De forma similar a como se hizo para el modelo SIR, se puede obtener la siguiente expresión, donde N es nuevamente la población inicial cerrada:

\begin{equation}
    S + E + I + R + Q + R + D + P = N
\end{equation}

El uso de este modelo en particular se discute más adelante.

\subsection{Modelo de Carga Potencial Máxima}
El objetivo de este modelo es seguir la evolución de la curva epidémica, y establecer predicciones a corto plazo $(t)$ de las cargas potenciales máximas de nuevos casos. El propósito de esto es estimar la saturación de un sistema de salud, lo que habilita una mejor toma de decisiones estratégicas de diverso tipo.

En este modelo, se establece que un nuevo caso tras un determinado tiempo $t$ es denominado como $C_t$, y existe un $I_t$ de personas infectadas, con $I_t = (C_t + C_{t-1})$. Estos casos, sin embargo, son puntos de transmisión de virus por hasta dos semanas, por lo que se puede establecer lo siguiente:

\begin{equation*}
    C_{t+1} \approx R_t (C_t + C_{t-1})
\end{equation*}

Esta expresión indica que todos los infectados en las dos semanas anteriores son contagiosos y van a contribuir a las infecciones del periodo $(t)$. Sin embargo, este no es el caso en la realidad, por lo que describir este proceso de transmisión en términos de probabilidad resulta más certero.

\begin{equation*}
    C_{t+1} =  \sum_{i=0}^{i=x} R_i p_i C_{tf - i} = R_t f(C_t + C_{t-1})
\end{equation*}

En esta expresión, $t$ es un periodo de tiempo, $i$ es una subdivisión de este periodo de tiempo, $tf$ es el último intervalo de este periodo, $p_i$ representa la probabilidad de que alguien infectado en el intervalo $tf-i$ infecte a alguien en el siguiente periodo de tiempo. Esto sugiere, entonces, una mejor expresión, tal que:

\begin{equation}
    C_{t+1} \approx fR_t (C_t + C_{t-1})
    \label{cpm_uno}
\end{equation}

En esta nueva expresión, $f$ corresponde a un factor de corrección como consecuencia de la distribución de probabilidad del intervalo serial. Para una carga máxima, este factor de corrección se estima como el máximo $p_i$.

\begin{figure}
    \includegraphics[width=\columnwidth]{cpm resultados.jpg}
    \caption{Resultados del uso del Modelo en el documento de Mauricio Canales y contribuidores. Fuente: \cite{canals_cuadrado_canals_2021}}
    \label{cpm resultados}
\end{figure}

Para el caso de Chile, los autores Mauricio Canales, Cristobal Cuadrado y Andrea Canales, desarrollaron este modelo con un periodo de tiempo $t$ equivalente a 1 semana. En este caso, el valor de $i$ corresponde a un día, y el valor de $x$ que se utiliza correspondería a 13, pues es la cantidad de tiempo que la persona enferma puede infectar, o sea, 2 semanas.

El desarrollo de este, se hizo con una distribución $\gamma$ con una media de 5 y una desviación estándar de $4,3$ días, por lo que su factor de corrección sería de $f \approx 0,8$. Bajo esto, generaron un modelo relativamente relacionado a la serie de Fibonacci, con dos valores iniciales. Para estos valores se utilizó $t_1 = 10$ y $t_2 = 65$, correspondientes a los casos nuevos en las primeras dos semanas de la epidemia.

Considerando que al rededor de un $3,5\%$ de los casos activos requiere tratamiento en una unidad de cuidados intensivos, y que estas suelen tener un estado de 'ocupadas' de dos semanas, y que la latencia entre el inicio de síntomas y el requerir una cama es de 1 semana, entregaron una expresión relacionada a la equación \eqref{cpm_uno} para predecir la cantidad de camas UCI ocupadas $U_t$:

\begin{equation}
    U_{t+1} = 0,035 (C_t + C_{t-1})
\end{equation}

En la figura \ref{cpm resultados} se observa los resultados de este modelo empleado por los autores, donde, pese a mantenerse certero en un inicio, termina sobre estimando los resultados de todos los parámetros que se intentaron predecir.
\subsection{Modelo de Gompertz}
El modelo de Gompertz, también llamado curva de Gompertz o función de Gompertz, es un modelo matemático para una serie temporal. La función es de tipo sigmoidea, y describe un crecimiento lengo en un periodo inicial y final, con uno más explosivo en un punto medio de la misma.

Inicialmente fue creada por Benjamin Gompertz para detallar su ley de la mortalidad humana, que se basa en un supuesto de que la resistencia de una persona a la muerte disminuye a medida que aumentan sus años, y fue descrito de la siguiente manera:

\begin{equation}
    N(t) = N(0) exp( -c ( exp(at) - 1))
\end{equation}

En esta ecuación, $N(t)$ representa un numero de individuos en un tiempo determinado $t$, por lo mismo, $N(0)$ representa la población inicial. $a$ denota una asíntota, $b$ y $c$ son valores positivos, que representan el desplazamiento a través del eje de las abcisas y la tasa de crecimiento, respectivamente. $exp$ denota la función exponencial.

\begin{figure*}
    \includegraphics[width=\textwidth]{casos diarios españa.png}
    \caption{Casos diarios de españa entre Marzo y Julio de 2020, fuente: cnecovid.isciii.es}
    \label{diarios españa}
\end{figure*}

En la publicación hecha por Sánchez-Villegas y colaboradores\cite{sanchez-villegas_codina_2020}, creada durante el año 2020, ellos decidieron utilizar el modelo de Gompertz para hacer una predicción del comportamiento de la epidemia de COVID-19 en España. Utilizando los datos de casos confirmados durante 47 días, generaron una curva de Gompertz que se ajustara lo más posible a sus datos.

¿Fue realmente acertado? Si bien, con la perspectiva inicial de dos años a futuro con respecto de esta publicación no lo parece para nada, contextualizando un poco a los datos que se tenían en ese momento podemos notar que la predicción fue, de hecho, relativamente acertada. En la figura \ref{diarios españa} podemos observar los casos diarios que se observan a lo largo del país entre las fechas de Marzo de 2020 hasta Julio del mismo año. La predicción inicial hecha por Sánchez-Villegas y colaboradores solo contemplaba datos hasta el 30 de abril de ese año, y contaba con solo 47 días de información, sin embargo, lograron demostrar que el pico de infección de españa llegó, efectivamente, en marzo de ese año, y que los contagios diarios iban extinguiendose.

Por desgracia, esta predicción cayó a mitades de Julio de 2020 y otras veces a futuro con la aparición de nuevas variantes y nuevos picos de infección en España y otros países del mundo. Aún así, el uso de este modelo predictivo en España y, según el autor, en la provincia de Hubei, donde ocurrió el primer caso de COVID, probó ser acertado para el corto plazo (Escala de Meses).

\begin{figure}
    \includegraphics[width=\columnwidth]{gompertz en santiago.jpg}
    \caption{Modelo de Gompertz para Santiago. Fuente: \cite{canals_cuadrado_canals_2021}}
    \label{gompertz santiago}
\end{figure}

Para el caso de Santiago, el autor Canales y colaboradores\cite{canals_cuadrado_canals_2021}, han creado un modelo de Gompertz en el punto de claro aplanamiento de la curva de casos totales en la región, tal como muestra la figura \ref{gompertz santiago}. Este gráfico es relativamente tardío para poder hacer predicciones, sin embargo muestra como un modelo exitoso si podría haber dado resultados correctos si se hubiera encontrado antes.

Al mirar ambos gráficos se observa como se apegan ambos totalmente a lo esperado. El gran problema que se tiene entonces, es como llegar a este modelo cuando los datos que se tienen son insuficientes, y esta es la gran dicotomía que se encuentra cuando se trabaja con modelos de predicción: solo en retrospectiva se pueden validar.
