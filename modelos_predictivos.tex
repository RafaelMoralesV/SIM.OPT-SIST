\section{Modelos predictivos}

La modelación matemática del virus se ha hecho desde el inicio de la pandemia. Ha ayudado a la predicción de la transmisión del virus, y comportamiento del mismo en poblaciones concretas. Por supuesto, estas simulaciones son tan solo tan buenas como la cantidad de estudio que se ha hecho del sistema en que se basan, y en el poco tiempo inicial en que salieron los primeros papers, las predicciones se caracterizaron por ser agridulcemente optimistas al mirarlas en la situación actual de la pandemia.

\begin{figure}
    \includegraphics[width=\columnwidth]{casos estimados españa.png}
    \caption{Casos estimados en España según el modelo del día 19 de abril (día 47 de la serie de datos registrados). Fuente: Modelos Predictivos de la Epidemia de Covid-19 en españa con curvas de gompertz, por Sánchez-Villegas, Pablo y Codina, Antonio Daponte}
\end{figure}

Según uno de los primeros reportes hechos en España, la expectativa de la pandemia era de un total de 240.000 contagios y 25.000 fallecidos, con un final pronosticado para la epidemia entre junio y julio de 2020\cite{sanchez-villegas_codina_2020}. La realidad de la situación de España es de 11.8 Millones de casos, y 104.000 muertes, y a fecha de abril de 2022, no parece estar muy cercano el final de esta epidemia, con nuevas variantes apareciendo aún.

Sin embargo, por más que sea divertido ver las predicciones de modelos poco entrenados en retrospectiva, estos eventualmente si lograron llegar a predicciones más y más acertadas, y gran parte del buen desarrollo de la pandemia se debe al uso que han tenido estos modelos matemáticos para poder predecir y anticipar la forma en que el virus se transmitiría o evolucionaría.

\subsection{Modelo SIR} 
Este es un modelo clásico para epidemias, de nombre SIR, o Susceptibles. Infectados y Recuperados, creado por Kermack y McKendrick. 

Se basa en el uso de ecuaciones diferenciales ordinarias para describir una mecánica de contagios en una población cerrada de $N$ individuos susceptibles a contagiarse por el virus. A partir de un contagio inicial, este modelo describe el contagio a una determinada velocidad de infección $I$. Tras un periodo de tiempo, una persona infectada en este modelo, que no haya fallecido, se vuelve inmune al mismo; deja de recibir nuevas infecciones y pasa a ser catalogado como Recuperado $R$. Con forme pasa el tiempo en esta simulación, la población que es susceptible al contagio disminuye, hasta el punto en que esta deja de existir, resultando en una transformación total de la población inicial susceptible en población resistente y personas fallecidas.

El sustento del modelo SIR son un trio de ecuaciones diferenciales ordinarias descritas de la siguiente manera:

\begin{equation} \frac{dS}{dt} = -\beta S I \label{SIR_1} \end{equation}

Describe el cambio de personas susceptibles en un instante en concreto. Este siempre debe tender a un valor menor o igual a cero, pues se espera que conforme avance la epidemia, menos personas susceptibles queden, pues van siendo infectadas en el trascurso de esta.

\begin{equation} \frac{dI}{dt} = \beta S I - \nu I \label{SIR_2} \end{equation}

Representa cuantos nuevos infectados aparecen en un instante. Este valor siempre es mayor o igual a cero.

\begin{equation} \frac{dR}{dt} = \nu I \label{SIR_3} \end{equation}

Representa la cantidad de nuevos individuos resistentes a la infección en un instante.

Para que estas expresiones funcionen, los valores de la razón de transmición $ \beta > 0 $ y la taza de recuperación $ \nu > 0 $.
La Expresión $ \beta S I $ corresponde a la cantidad de nuevas infecciones en un determinado instante. Si reemplazamos la ecuación \eqref{SIR_3} en \eqref{SIR_2} podemos conseguir un valor para esta expresión en base a la derivada de Infectados y Recuperados tal que:

\begin{equation} \frac{dI}{dt} + \frac{dR}{dt} = \beta S I \end{equation}

La cual, a su vez, que posible reemplazar dentro de la ecuación \eqref{SIR_1}.

\begin{equation} \frac{dS}{dt} + \frac{dI}{dt} + \frac{dR}{dt} = 0 \end{equation}

Integrar esta última expresión nos da una expresión base para un modelo SIR que muestra uno de los problemas que contiene:

\begin{equation} S + I + R = Cte \end{equation}

El valor constante que se encontró con estas operaciones corresponde al numero $N$ del que se habló en un inicio. Como la población es cerrada, nunca cambia a lo largo del tiempo. El modelo SIR asume, además, que la epidemia presentada es relativamente breve en el tiempo. Tampoco ocurren nacimientos ni muertes naturales. No hay un periodo de infección latente, lo que implica que un individuo se vuelve infeccioso en el instante en que es infectado él mismo. La inmunidad que se obtiene del virus es permanente, y, el mayor problema, una mezcla en masa de individuos.

La mezcla en masa de individuos asume que la razón de encuentro entre la población susceptible e infectada es proporcional al producto de ambas poblaciones. Si se dobla la cantidad de cualquiera de estas resulta en el doble de infecciones en un instante de tiempo, lo cual es una suposición extraña. Un individuo en concreto solo mantiene contacto con una cantidad reducidad de otros individuos dentro de su propia comunidad.



\begin{figure}
    \includegraphics[width=\columnwidth]{Modelo SIR.png}
    \caption{Grafica de un Modelo SIR de un problema concreto \cite{weiss_2013}}
\end{figure}

\subsection{Otros modelos}
La segunda forma de modelado de la pandemia es con modelos no clásicos de diverso estilo. Estos son modelos predictivos basados en en uso de datos cronologicos del COVID para la generación de modelos estadísticos más tradicionales, como modelos basados en programas de computación hechos a base de Machine Learning. Estos modelos suelen tener una complejidad alta, por lo que no se explicarán en este documento.