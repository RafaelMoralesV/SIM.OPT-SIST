% Este archivo gobierna la primera sección correspondiente únicamente
% a la introducción.

\section{Introduccion}
A fecha de 31 de diciembre de 2019, los primeros casos humanos del virus COVID-19 son confirmados en la ciudad de Wuhan, provincia de Hubei, en la república popular china. La Organización Mundial de la Salud (OMS) lo declaró como una Emergencia que debería ser de preocupación internacional el 30 de Enero de 2020, y finalmente como Pandemia el 11 de Marzo de ese mismo año \cite{organization_2020}. A estas alturas de la Pandemia, más de 2 años después, aún no es posible determinar de forma contundente como es que los humanos de China Continental fueron inicialmente o previamente infectados con el Virus SARs-CoV-2, y reportes anteriores nombraban la aparición este virus en Europa tan temprano como Octubre de 2019.\cite{sakay_2021}

Durante noviembre de 2020, Estados Unidos fue el primer país en tener al menos 10 millones de casos confirmados\cite{stein_2020}, seguido, en diciembre del mismo año, por India\cite{asrar_2020}, Brazil en febrero del 2021\cite{leite_2021}, Reino Unido en Noviembre de 2021\cite{standard_2021}, Rusia en Diciembre de ese año\cite{asia_2021}, Francia a inicios del 2022\cite{television_2022}, Turquía una semana después, luego Italia, y a inicios de Febrero Alemania y España, Corea del Sur durante Marzo del 2022, días después de sobrepasar a Japón, y finalmente, en Abril, Vietnam. A fecha actual, estos son los únicos 12 países que tienen al menos diez millones de casos confirmados.

No es necesario agregar, a estas alturas, que el virus SARs-CoV-2 es uno áltamente infeccioso, característica que ha abierto ya la discusión de si alguna vez será erradicado, o si inevitablemente será una endemia con la que viviremos para siempre.

Este virus es un clado dentro de la familia de los Coronaviridae, una especie de Virus SARs, que son relacionados con el síndrome respiratorio agudo severo o grave). Guarda un parecido con el visrus SARs-CoV-1, que fue detectado en un inicio el 2003, sin embargo es mucho más similar a otros tipos de coronavirus que infecta únicamente animales\cite{rehman_shafique_ihsan_liu_2020}.

Como todo virus, la hiper multiplicación de su periodo 'de vida' ha ido generando mutaciones pequeñas a su genoma, que han sido reconocidos por la comunidad cientifica como linajes o variantes del virus original distintos del mismo. Por suerte, sabemos que el SARs-CoV-2 no parece mutar en exceso como otros virus, sin embargo, varias de estas cepas han reactivado periodos de infección en diversos países. Actualmente, la más vigente a fecha de redacción es la variante Omicron, que es reconocida como una, sino la variante más infecciosa hasta ahora, pese a ser una variante menos dañina, o con menor tendencia a hospitalización entre infectados.\cite{unicef_2022}

El estudio de este virus ha sido constante e incesante. La fabricación de las primeras vacunas llegaron en tiempo record con respecto a cualquier otro virus. Este esfuerzo ha logrado reconocer diversas características de este virus, como la proteína que utiliza para penetrar en células humanas. Esta recopilación constante de datos ha permitido a científicos de todo el mundo la generación de diversos modelos para la simulación de infecciones, comportamiento en poblaciones humanas y predicciones de la pandemia de diversos tipos. A continuación se hará un estudio y explicación de como estos modelos y simulaciones han servido como una herramienta indispensable para la supervivencia a la pandemia.