% Este archivo gobierna la primera sección correspondiente únicamente
% a la introducción.

\section{Introduccion}
A fecha de 31 de diciembre de 2019, los primeros casos humanos del virus COVID-19 son confirmados en la ciudad de Wuhan, provincia de Hubei, en la república popular china. La Organización Mundial de la Salud (OMS) lo declaró como una Emergencia que debería ser de preocupación internacional el 30 de Enero de 2020, y finalmente como Pandemia el 11 de Marzo de ese mismo año \cite{organization_2020}. A estas alturas de la Pandemia, más de 2 años después, aún no es posible determinar de forma contundente como es que los humanos de China Continental fueron inicialmente o previamente infectados con el Virus SARS-CoV-2, y reportes anteriores nombraban la aparición este virus en Europa tan temprano como Octubre de 2019.\cite{sakay_2021}

Durante noviembre de 2020, Estados Unidos fue el primer país en tener al menos 10 millones de casos confirmados\cite{stein_2020}, seguido, en diciembre del mismo año, por India\cite{asrar_2020}, Brazil en febrero del 2021\cite{leite_2021}, Reino Unido en Noviembre de 2021\cite{standard_2021}, Rusia en Diciembre de ese año\cite{asia_2021}, Francia a inicios del 2022\cite{television_2022}, Turquía una semana después, luego Italia, y a inicios de Febrero Alemania y España, Corea del Sur durante Marzo del 2022, días después de sobrepasar a Japón, y finalmente, en Abril, Vietnam. A fecha actual, estos son los únicos 12 países que tienen al menos diez millones de casos confirmados.