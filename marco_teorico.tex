\section{Marco teórico}
Se ofrecen, entonces, las siguientes definiciones como sustento o marco teórico para este documento:

%%%%%%%%%%%%%%%%%%%%%%%%%%%%%%%%%%%%%%%%%%%%%%%%%%%%%%%%%%%%%%%%%%%%%%%%%%%%%%%%%%%%%

\subsection{Modelo}
Es una representación abstracta de un sistema, utilizada comúnmente para el estudio del mismo. Para efectos de este documento, nos referiremos específicamente a Modelos de tipo Matemáticos, o séa, representaciones matemáticas de sistemas.

%%%%%%%%%%%%%%%%%%%%%%%%%%%%%%%%%%%%%%%%%%%%%%%%%%%%%%%%%%%%%%%%%%%%%%%%%%%%%%%%%%%%%

\subsection{Sistema}
Conjunto de elementos que interactúan entre sí.

Conjunto de reglas o principios sobre una materia racionalmente enlazados entre sí.

Conjunto de 'cosas' que, relacionadas entre sí de forma ordenada, contribuyen a determinado objeto.

%%%%%%%%%%%%%%%%%%%%%%%%%%%%%%%%%%%%%%%%%%%%%%%%%%%%%%%%%%%%%%%%%%%%%%%%%%%%%%%%%%%%%

\subsection{Sistema abierto}
Es un sistema que interactúa con su entorno.

%%%%%%%%%%%%%%%%%%%%%%%%%%%%%%%%%%%%%%%%%%%%%%%%%%%%%%%%%%%%%%%%%%%%%%%%%%%%%%%%%%%%%

\subsection{Sistema cerrado}
Es un sistema que no interactúa con su entorno.

%%%%%%%%%%%%%%%%%%%%%%%%%%%%%%%%%%%%%%%%%%%%%%%%%%%%%%%%%%%%%%%%%%%%%%%%%%%%%%%%%%%%%

\subsection{Subsistema}
Los sistemas pueden ser compuestos por otros sistemas menores o más pequeños, que en interación de unos con otros producen el sistema mayor.

%%%%%%%%%%%%%%%%%%%%%%%%%%%%%%%%%%%%%%%%%%%%%%%%%%%%%%%%%%%%%%%%%%%%%%%%%%%%%%%%%%%%%

\subsection{Modelo determinístico}
Es un modelo cuyas relaciones siempre producen el mismo comportamiento cuando reciben la misma 'entrada'. Esto quiere decir que el sistema no tiene elementos aleatoreos que lo componen. En este tipo de modelos, las variables internas y de salida quedan determinadas cuando se definan las variables de entrada, parámetros y variables de estado, por lo que las relaciones funcionales entre las mismas están siempre bien definidas.

%%%%%%%%%%%%%%%%%%%%%%%%%%%%%%%%%%%%%%%%%%%%%%%%%%%%%%%%%%%%%%%%%%%%%%%%%%%%%%%%%%%%%

\subsection{Modelo estocástico}
Es un modelo donde una o más relaciones está basada en elementos aleatóreos, lo que implica comportamientos múltiples bajo una misma entrada para el modelo. La idea de estos modelos es representar un sistema con un comportamiento más caótico, como una máquina tragamodenas: en este sistema, la misma entrada (colocar una moneda) genera resultados completamente distintos siempre, por lo que el modelo debe ser capaz de representar esto mediante el uso de aleatoreidad y resultados inciertos.

Es importante recalcar que, si un modelo determinístico es utilizado con entradas estocásticas, su comportamiento será equivalente al de un modelo estocástico.

%%%%%%%%%%%%%%%%%%%%%%%%%%%%%%%%%%%%%%%%%%%%%%%%%%%%%%%%%%%%%%%%%%%%%%%%%%%%%%%%%%%%%

\subsection{Simulación}
Consiste en estudiar el comportamiento de un sistema a través de el uso de un modelo que lo represente. Ambos deben comportarse de manera similar, de tal manera de que el comportamiento del modelo sea fiel a lo que se espera que sea el comportamiento del sistema dado los mismas entradas, y por lo mismo, la simulación depende totalmente de la calidad del modelo para ser exitosa.

%%%%%%%%%%%%%%%%%%%%%%%%%%%%%%%%%%%%%%%%%%%%%%%%%%%%%%%%%%%%%%%%%%%%%%%%%%%%%%%%%%%%%

\subsection{Simulación continua}
Es aquella simulación en el que el estado del modelo cambia permanentemente en el tiempo. Ocurre cuando las relaciones funcionales entre las variables del sistema solo permiten que el estado se transforme en el tiempo de manera continua.

%%%%%%%%%%%%%%%%%%%%%%%%%%%%%%%%%%%%%%%%%%%%%%%%%%%%%%%%%%%%%%%%%%%%%%%%%%%%%%%%%%%%%

\subsection{Simulación discreta}
Es aquella simulación en el que el cambio de estado se produce cada cierto intervalo de tiempo. Los modelos de este tipo se caracterizan porque las variables cambian únicamente en un instante determinado o secuencia de instantes, y permanecen constantes el resto del tiempo.

%%%%%%%%%%%%%%%%%%%%%%%%%%%%%%%%%%%%%%%%%%%%%%%%%%%%%%%%%%%%%%%%%%%%%%%%%%%%%%%%%%%%%

\subsection{Flujo}
"Corriente", o "ir de un lado a otro". En general, el término de flujo se utiliza cuando se hace referencia al movimiento de algo.

%%%%%%%%%%%%%%%%%%%%%%%%%%%%%%%%%%%%%%%%%%%%%%%%%%%%%%%%%%%%%%%%%%%%%%%%%%%%%%%%%%%%%

\subsection{Software}
Conjunto de programas, instrucciones y reglas informáticas para ejecutar tareas en una computadora.

%%%%%%%%%%%%%%%%%%%%%%%%%%%%%%%%%%%%%%%%%%%%%%%%%%%%%%%%%%%%%%%%%%%%%%%%%%%%%%%%%%%%%

\subsection{Hardware}
Conjunto de elementos físicos o materiales que constituyen una máquina. Para el contexto, se refiere particularmente a los que componene a una computadora o sistema informático.

%%%%%%%%%%%%%%%%%%%%%%%%%%%%%%%%%%%%%%%%%%%%%%%%%%%%%%%%%%%%%%%%%%%%%%%%%%%%%%%%%%%%%
