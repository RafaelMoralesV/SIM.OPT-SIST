\section{Marco teórico}
Se ofrecen a continuación, las siguientes definiciones como sustento o marco teórico para este documento:

%%%%%%%%%%%%%%%%%%%%%%%%%%%%%%%%%%%%%%%%%%%%%%%%%%%%%%%%%%%%%%%%%%%%%%%%%%%%%%%%%%%%%

\subsection{Modelo}
Es una representación abstracta de un sistema, utilizada comúnmente para el estudio del mismo. Para efectos de este documento, nos referiremos específicamente a Modelos de tipo Matemáticos, o séa, representaciones matemáticas de sistemas.

%%%%%%%%%%%%%%%%%%%%%%%%%%%%%%%%%%%%%%%%%%%%%%%%%%%%%%%%%%%%%%%%%%%%%%%%%%%%%%%%%%%%%

\subsection{Sistema}
Conjunto de elementos que interactúan entre sí.

Conjunto de reglas o principios sobre una materia racionalmente enlazados entre sí.

Conjunto de 'cosas' que, relacionadas entre sí de forma ordenada, contribuyen a determinado objeto.

%%%%%%%%%%%%%%%%%%%%%%%%%%%%%%%%%%%%%%%%%%%%%%%%%%%%%%%%%%%%%%%%%%%%%%%%%%%%%%%%%%%%%

\subsection{Sistema abierto}
Es un sistema que interactúa con su entorno.

%%%%%%%%%%%%%%%%%%%%%%%%%%%%%%%%%%%%%%%%%%%%%%%%%%%%%%%%%%%%%%%%%%%%%%%%%%%%%%%%%%%%%

\subsection{Sistema cerrado}
Es un sistema que no interactúa con su entorno.

%%%%%%%%%%%%%%%%%%%%%%%%%%%%%%%%%%%%%%%%%%%%%%%%%%%%%%%%%%%%%%%%%%%%%%%%%%%%%%%%%%%%%

\subsection{Subsistema}
Los sistemas pueden ser compuestos por otros sistemas menores o más pequeños, que en interación de unos con otros producen el sistema mayor.

%%%%%%%%%%%%%%%%%%%%%%%%%%%%%%%%%%%%%%%%%%%%%%%%%%%%%%%%%%%%%%%%%%%%%%%%%%%%%%%%%%%%%

\subsection{Modelo determinístico}
Es un modelo cuyas relaciones siempre producen el mismo comportamiento cuando reciben la misma 'entrada'. Esto quiere decir que el sistema no tiene elementos aleatoreos que lo componen. En este tipo de modelos, las variables internas y de salida quedan determinadas cuando se definan las variables de entrada, parámetros y variables de estado, por lo que las relaciones funcionales entre las mismas están siempre bien definidas.

%%%%%%%%%%%%%%%%%%%%%%%%%%%%%%%%%%%%%%%%%%%%%%%%%%%%%%%%%%%%%%%%%%%%%%%%%%%%%%%%%%%%%

\subsection{Modelo estocástico}
Es un modelo donde una o más relaciones está basada en elementos aleatóreos, lo que implica comportamientos múltiples bajo una misma entrada para el modelo. La idea de estos modelos es representar un sistema con un comportamiento más caótico, como una máquina tragamodenas: en este sistema, la misma entrada (colocar una moneda) genera resultados completamente distintos siempre, por lo que el modelo debe ser capaz de representar esto mediante el uso de aleatoreidad y resultados inciertos.

Es importante recalcar que, si un modelo determinístico es utilizado con entradas estocásticas, su comportamiento será equivalente al de un modelo estocástico.

%%%%%%%%%%%%%%%%%%%%%%%%%%%%%%%%%%%%%%%%%%%%%%%%%%%%%%%%%%%%%%%%%%%%%%%%%%%%%%%%%%%%%

\subsection{Ecuación}
En matemáticas: Igualdad entre dos o más expresiones matemáticas.

Igualdad o paralelismo entre dos o más cosas que pueden o no guardar relación.

%%%%%%%%%%%%%%%%%%%%%%%%%%%%%%%%%%%%%%%%%%%%%%%%%%%%%%%%%%%%%%%%%%%%%%%%%%%%%%%%%%%%%

\subsection{Función (Matemáticas)}
Relación que se establece entre dos conjuntos a través de la cual, para para cada elemento del primer conjunto, se le asigna uno o ningún elemento del segundo, o en otros términos, una entrada produce un único o ningún resultado.

%%%%%%%%%%%%%%%%%%%%%%%%%%%%%%%%%%%%%%%%%%%%%%%%%%%%%%%%%%%%%%%%%%%%%%%%%%%%%%%%%%%%%

\subsection{Derivada (Matemáticas)}
Resultado de un límite, que representa la pendiente de una recta tangente a la curva de la gráfica de una función en un punto concreto de la misma.

%%%%%%%%%%%%%%%%%%%%%%%%%%%%%%%%%%%%%%%%%%%%%%%%%%%%%%%%%%%%%%%%%%%%%%%%%%%%%%%%%%%%%

\section{Ecuación diferencial (Matemáticas)}

Ecuación matemática que relaciona, de manera no trivial, una función desconocida con respecto de una o más derivadas de esta función. Pueden depender de una o más variables dependientes o independientes, y bajo esto, pueden ser clasificadas como ecuaciones diferenciales "ordinarias" o "parciales".

%%%%%%%%%%%%%%%%%%%%%%%%%%%%%%%%%%%%%%%%%%%%%%%%%%%%%%%%%%%%%%%%%%%%%%%%%%%%%%%%%%%%%

\pagebreak